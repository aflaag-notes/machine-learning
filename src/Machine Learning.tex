\documentclass[a4paper, 12pt]{report}

\usepackage[dvipsnames]{xcolor}

%%%%%%%%%%%%%%%%%
% Set Variables %
%%%%%%%%%%%%%%%%%

\def\useItalian{0}  % 1 = Italian, 0 = English

\def\courseName{Machine Learning}

\def\coursePrerequisites{
    TODO
}

\def\book{TODO}

% \def\authorName{Simone Bianco}
% \def\email{bianco.simone@outlook.it}
% \def\github{https://github.com/Exyss/university-notes}
% \def\linkedin{https://www.linkedin.com/in/simone-bianco}

\def\authorName{Alessio Bandiera}
\def\email{alessio.bandiera02@gmail.com}
\def\github{https://github.com/aflaag-notes}
\def\linkedin{https://www.linkedin.com/in/alessio-bandiera-a53767223}

% Do not change

%%%%%%%%%%%%
% Packages %
%%%%%%%%%%%%

\usepackage{../../packages/Nyx/nyx-packages}
\usepackage{../../packages/Nyx/nyx-styles}
\usepackage{../../packages/Nyx/nyx-frames}
\usepackage{../../packages/Nyx/nyx-macros}
\usepackage{../../packages/Nyx/nyx-title}
\usepackage{../../packages/Nyx/nyx-intro}

%%%%%%%%%%%%%%
% Title-page %
%%%%%%%%%%%%%%

\logo{../../packages/Nyx/logo.png}

\ifx\useItalian0
    \institute{\curlyquotes{\hspace{0.25mm}Sapienza} Università di Roma}
    \faculty{Ingegneria dell'Informazione,\\Informatica e Statistica}
    \department{Dipartimento di Informatica}
    \subtitle{Appunti integrati con il libro \book}
    \author{\textit{Autore}\\\authorName}
\else
    \institute{\curlyquotes{\hspace{0.25mm}Sapienza} University of Rome}
    \faculty{Faculty of Information Engineering,\\Informatics and Statistics}
    \department{Department of Computer Science}
    \subtitle{Lecture notes integrated with the book \book}
    \author{\textit{Author}\\\authorName}
\fi

\title{\courseName}
\date{\today}

% \supervisor{Linus \textsc{Torvalds}}
% \context{Well, I was bored\ldots}

%%%%%%%%%%%%
% Document %
%%%%%%%%%%%%

\begin{document}
    \maketitle

    % The following style changes are valid only inside this scope 
    {
        \hypersetup{allcolors=black}
        \fancypagestyle{plain}{%
        \fancyhead{}        % clear all header fields
        \fancyfoot{}        % clear all header fields
        \fancyfoot[C]{\thepage}
        \renewcommand{\headrulewidth}{0pt}
        \renewcommand{\footrulewidth}{0pt}}

        \romantableofcontents
    }

    \introduction

    %%%%%%%%%%%%%%%%%%%%%
    
    \chapter{TODO}
    
    \section{Learning problems}

    \subsection{TODO}
    
    A \tbf{machine learning problem} is defined by the following \tit{three components}.

    \begin{frameddefn}{Learning}
        \tbf{Learning} is defined as \tit{improving}, through \tit{experience} $E$, at some \tit{task} $T$, with respect to a \tit{performance measure} $P$.
    \end{frameddefn}

    \begin{example}[Machine Learning problem]
        Consider the problem of learning how to play \href{https://en.wikipedia.org/wiki/Checkers}{Checkers}; in this example, the \tit{task} $T$ is to be able to play the game itself, the \tit{performance measure} $P$ could be the percentage of games won in a tournament, but \tit{experience} $E$ is more complex.
    \end{example}

    In general, \tit{experience} can be acquired in several ways:

    \begin{itemize}
        \item in this example, a human expert may suggest optimal moves for each configuration of the board; however, this approach may not generalize for any problem, as human experts may not exists for certain tasks;
        \item alternatively, the computer may play against a human, and automatically detect win, draw and loss configurations;
        \item lastly, the computer may play against itself, learning from its own successes and failures.
    \end{itemize}

    For this particular game, a possible \tbf{target function} (the function that would be useful to learn in order to solve the learning problem) could be the following $$\func{\mathrm{ChooseMove}}{\mathrm{Board}}{\mathrm{Move}}$$ which, given a board state, returns the best move to perform, but also $$\func{V}{\mathrm{Board}}{\R}$$ which assigns a \tit{score} to a given board.

    For instance, consider the following target function: $$\hat V(b) = w_0 + w_1 \cdot bp(b) + w_2 \cdot rp(b) + w_3 \cdot bk(b) + w_4 \cdot rk(b) + w_5 \cdot bt(b) + w_6 \cdot rt(b)$$ where $b$ is a given \tit{board state}, and

    \begin{itemize}
        \item $bp(b)$ is the number of \tit{black pieces}
        \item $rp(b)$ is the number of \tit{red pieces}
        \item $bk(b)$ is the number of \tit{black kings}
        \item $rk(b)$ is the number of \tit{red kings}
        \item $bt(b)$ is the number of \tit{red pieces threatened by black pieces}
        \item $rt(b)$ is the number of \tit{black pieces threatened by red pieces}
    \end{itemize}

    In this formulation, $\hat V$ is a \tit{linear compbination} of multiple coefficients $w_i$, which are unknown. Therefore, in this example \tbf{goal} of the \tit{learning problem} is to \tbf{learn $\hat V$}, or equivalently, to \tbf{estimate each coefficient $w_i$}. Note that this function \tit{can be computed}.
    
    \begin{frameddefn}{Dataset}
        Let $V(b)$ be the \tit{true target function} (always \tit{unknown}), $\hat V(b)$ be the \tit{learned function} --- an approximation of $V(b)$ computed by the \tit{learning algorithm} --- and $V_t(b)$ the \tit{training value} of $b$ in the \tit{training data}. A \tbf{dataset} is a set of samples, denoted as $$D = \{(b_i, V_t(b_i)) \mid i \in [1, n]\}$$
    \end{frameddefn}

\end{document}
